\documentclass[11pt,a4paper,sans,french]{moderncv}
\usepackage[utf8]{inputenc}
\usepackage{babel}

% thiner margins
\usepackage[margin=1.8cm,bottom=2.4cm]{geometry}
\recomputelengths{}

% skill levels
\usepackage{tikz}
\newcommand{\skilllevel}[1]{\tikz [x=1pt, y=.666em, xslant=.3, rounded corners] {% The devil is in the detail
  \fill [gray!15] (0, 0) rectangle (100, 1);%
  \fill [color1] (0, 0) rectangle (100 * #1, 1);%
}}
\newcommand{\Expert}{\skilllevel{1}}
\newcommand{\Avance}{\skilllevel{0.75}}
\newcommand{\Autonome}{\skilllevel{0.5}}
\newcommand{\Notions}{\skilllevel{0.25}}

\moderncvstyle{casual}
\moderncvcolor{green}

\name{Romain}{\color{color1}\textsc{Déoux}}
\title{Analyste Développeur}
\phone[mobile]{06~82~87~38~02}
\email{romain.deoux@gmail.com}
\social[github]{quarthex}
% I should take a shot, maybe…
\photo{2017-12-29.jpg}

\nopagenumbers{}

\begin{document}
\maketitle

\section{Compétences}

\begin{tabular}{p{\hintscolumnwidth} @{\hspace{\separatorcolumnwidth}} p{4cm} r}
 \raggedleft\hintstyle{Application}
  & C            & \Expert{}   \\
  & C++          & \Expert{}   \\
  & Rust         & \Notions{}  \\
  & Java SE      & \Notions{}  \\
  & C\#          & \Notions{}  \\
  & Haskell      & \Notions{}  \\[1ex]
 \raggedleft\hintstyle{Script}
  & Korn shell   & \Avance{}   \\
  & Python3      & \Avance{}   \\
  & Perl         & \Autonome{} \\[1ex]
 \raggedleft\hintstyle{Web}
  & Javascript   & \Avance{}   \\
  & NodeJS       & \Avance{}   \\
  & CSS3         & \Autonome{} \\
  & CoffeeScript & \Autonome{} \\
  & HTML5        & \Autonome{} \\
  & PHP7         & \Autonome{} \\
  & SQL          & \Autonome{} \\
  & Sass         & \Autonome{}
\end{tabular}

\bigskip{}

\cvline{Linux}{Maîtrise de l’environnement GNU/Linux}
\cvline{Workflow}{Méthodes Agiles, JIRA, Mantis, Test Driven Development}
\cvline{CI}{Buildbot, Docker, Jenkins (maître/esclaves), VirtualBox}
\cvline{Versionnage}{Subversion, git}
\cvline{Build system}{Grunt, CMake, GNU Makefile}
\cvline{Embarqué}{Buildroot, BusyBox, Windows CE}

\medskip{}

\cvline{Langue}{Anglais courant}

\section{Dipl\^{o}mes}

\cventry{2009--2010}{Licence professionnel}{UFR des Sciences et Techniques}{Besançon}{}{%
 Systèmes Informatiques et Logiciels
 spécialité Conception et Développement Orienté Objet d'Applications Multi-tiers}

\cventry{2007--2009}{BTS}{Lycée Jules Haag}{Besançon}{}{%
 Informatique et Réseau pour l'Industrie et les Services techniques}

\cventry{2007}{Baccalauréat Scientifique}{Lycée Jules Haag}{Académie de Besançon}{}{%
 Sciences de l'Ingénieur spécialité Mathématiques}

\clearpage
\section{Parcours professionnel}

\cventry{Depuis 2010}{Technicien analyste programmeur}
{\href{https://www.aprogsys.com/}{Aprogsys}}{Besançon}{Salarié}{%
 \begin{itemize}
  \item Sous-traitance pour \href{http://vixtechnology.com/}{VIX}
        \begin{itemize}
         \item Développement d'un système billettique sur Linux embarqué
               en C99 et C++11
         \item Développement sur systèmes embarqués (pupitre valideur et
               valideur de titres de transport, concentrateur de données)
         \item Décodage/encodage de cartes sans contact
         \item Travail en collaboration avec des ingénieurs à l'étranger
         \item Implémentation de tests fonctionnels automatisés
         \item Développement d'une application web embarquée de configuration
               de carte sans contact.
               Interface utilisant Bootstrap et jQuery compilée via Grunt.
               Serveur asynchrone (bibliothèque libuv) en C communiquant avec
               un backend via le bus D-Bus.
        \end{itemize}
  \item Sous-traitance pour \href{http://flowbird.group/}{Flowbird}
        \begin{itemize}
         \item Développement sur systèmes embarqués (imprimante de tickets,
               carte électronique de surveillance, dépilleur de cartes)
         \item Décodage/encodage de pistes magnétiques
         \item Programmation par interruptions
         \item Protocole de communication série (basé sur HDLC)
         \item Recettes de construction automatique de fichiers CMake
         \item Intégration Continue via Jenkins
         \item Tests fonctionnels semi-automatiques
        \end{itemize}
  \item Analyse et développement logiciel spécialisé dans le domaine
        de la traçabilité et du monitoring via la technologie RFID
        \begin{itemize}
         \item \href{http://www.caves-explorer.com/}{Caves Explorer}~:
               Logiciel de gestion de caves à vin grand public assisté par des
               radio-étiquettes SLI, développé à l'aide du framework Qt.
               Synchronisation des données entre une base de données distante
               MySQL et une base locale SQLite.
               Solutions dérivés intégrés à la solution principale.
         \item \href{http://www.beetracking.com/}{Beetracking}~:
               Solution de suivi de rucher et de marquage de ruche par
               radio-étiquettes passives {UHF}.
               Développement embarqué sur BOLYMIN, Armadeus et terminaux
               portatifs durcis {PSION}.
               Envois de SMS (commandes AT).
         \item
               Développement d'une solution de traçabilité de chaîne de
               production.
               Système d'informations distribués.
               Radio-étiquettes {MIFARE}.
        \end{itemize}
  \item Développement d'une IHM en C/GTK pour gérer des données vectorielles au
        sein d'une carte dans le cadre d'un projet destiné à la Défense.
  \item Participation au développement d'applications Web.
 \end{itemize}}

\cventry{2010\\\small\color{gray}{2 mois}}{Ouvier saisonnier}
{R. Bourgeois}{Besançon}{Salarié}{Cerclage de palette}

\cventry{2010\\\small\color{gray}{3 mois}}{Stage en entreprise}
{Parkeon (Aujourd'hui appelé Flowbird)}{Besançon}{Stagiaire}{%
 Programmation en Java d'un logiciel chargé de piloter un banc d'encodage de
 cartes magnétiques et de gérer les mappings de cartes}

\cventry{2009\\\small\color{gray}{2 mois}}{Ouvrier d'exécution}
{Renaissance SARL}{Saint Nazaire (Loire Atlantique)}{Salarié}{%
 Travaux de rénovation, maroufflage de moteur de bateau}

\cventry{2008\\\small\color{gray}{6 semaines}}{Stage en entreprise}
{Selma (Aujourd'hui appelé \href{http://www.safed.ch/}{SAFED})}{Besançon}{Stagiaire}{%
 Logiciel de monitoring de température de fours industriels}

\section{Centres d'intérêt}

\cvlistitem{Aikido}
\cvlistitem{Apiculture}

\end{document}
