\documentclass[12pt,a4paper,sans]{moderncv}
\usepackage[utf8]{inputenc}

% thiner margins
\usepackage[margin=1.8cm,bottom=2.4cm]{geometry} 
\recomputelengths{}

% skill levels
\usepackage[heightr=0.75,roundnessr=0,width=8em,borderwidth=0pt,ticksheight=0pt,linecolor=white,filledcolor=gray]{progressbar}
\newcommand{\Expert}{\progressbar{1}}
\newcommand{\Avance}{\progressbar{0.75}}
\newcommand{\Autonome}{\progressbar{0.5}}
\newcommand{\Notions}{\progressbar{0.25}}

\moderncvstyle{casual}
\moderncvcolor{green}

\name{Romain}{\color{color1}\textsc{Déoux}}
\title{Analyste Développeur}
\phone[mobile]{06~82~87~38~02}
\email{romain.deoux@gmail.com}
\social[github]{quarthex}
% I should take a shot, maybe…
%\photo[64pt]{}

\nopagenumbers{}

\begin{document}
\maketitle

\section{Compétences}

\begin{tabular}{p{\hintscolumnwidth} @{\hspace{\separatorcolumnwidth}} p{4cm} r}
	\raggedleft\hintstyle{Application}
	  & C            & \Expert{}   \\
	  & C++          & \Expert{}   \\
	  & Rust         & \Notions{}  \\
	  & Java SE      & \Notions{}  \\
	  & C\#          & \Notions{}  \\
	  & Haskell      & \Notions{}  \\[1ex]
	\raggedleft\hintstyle{Script}
	  & Korn shell   & \Avance{}   \\
	  & Python3      & \Avance{}   \\
	  & Perl         & \Autonome{} \\[1ex]
	\raggedleft\hintstyle{Web}
	  & Javascript   & \Avance{}   \\
	  & NodeJS       & \Avance{}   \\
	  & CSS3         & \Autonome{} \\
	  & CoffeeScript & \Autonome{} \\
	  & HTML5        & \Autonome{} \\
	  & PHP7         & \Autonome{} \\
	  & SQL          & \Autonome{} \\
	  & Sass         & \Autonome{} 
\end{tabular}

\bigskip{}

\cvline{Linux}{Maîtrise de l’environnement GNU/Linux}
\cvline{Workflow}{Méthodes Agiles, JIRA, Mantis, Test Driven Development}
\cvline{CI}{Buildbot, Docker, Jenkins (maître/esclaves), VirtualBox}
\cvline{Versionnage}{Subversion, git}
\cvline{Build system}{Grunt, CMake, GNU Makefile}
\cvline{Embarqué}{Buildroot, BusyBox, Windows CE}

\medskip{}

\cvline{Langue}{Anglais courant}

\section{Dipl\^{o}mes}

\cventry{2009--2010}{Licence professionnel}{UFR des Sciences et Techniques}{Besançon}{}{%
	Systèmes Informatiques et Logiciels
spécialité Conception et Développement Orienté Objet d'Applications Multi-tiers}

\cventry{2007--2009}{BTS}{Lycée Jules Haag}{Besançon}{}{%
Informatique et Réseau pour l'Industrie et les Services techniques}

\cventry{2007}{Baccalauréat Scientifique}{Lycée Jules Haag}{Académie de Besançon}{}{%
Sciences de l'Ingénieur spécialité Mathématiques}

\clearpage
\section{Parcours professionnel}

\cventry{Depuis 2010}{Technicien analyste programmeur}{\href{https://www.aprogsys.com/}{Aprogsys}}{Besançon}{Salarié}{%
	\begin{itemize}
		\item Sous-traitance pour \href{http://vixtechnology.com/}{VIX}
		      \begin{itemize}
		      	\item Développement d'un système billettique sur Linux embarqué
		      	      en C99 et C++11
		      	\item Développement sur systèmes embarqués (pupitre valideur et
		      	      valideur de titres de transport, concentrateur de données)
		      	\item Décodage/encodage de cartes sans contact
		      	\item Travail en collaboration avec des ingénieurs à l'étranger
		      	\item Implémentation de tests fonctionnels automatisés
		      	\item Développement d'une application web embarquée de
		      	      configuration de carte sans contact.
		      	      Interface utilisant Bootstrap et jQuery compilée via
		      	      Grunt.
		      	      Serveur asynchrone (bibliothèque libuv) en C communiquant
		      	      avec un backend via le bus D-Bus.
		      \end{itemize}
		\item Sous-traitance pour \href{http://www.parkeon.fr/}{Parkeon}
		      \begin{itemize}
		      	\item Développement sur systèmes embarqués (imprimante de
		      	      tickets, carte électronique de surveillance, dépilleur de
		      	      cartes)
		      	\item Décodage/encodage de pistes magnétiques
		      	\item Programmation par interruptions
		      	\item Protocole de communication série (basé sur HDLC)
		      	\item Recettes de construction automatique de fichiers CMake
		      	\item Intégration Continue via Jenkins
		      \end{itemize}
		\item Analyse et développement logiciel spécialisé dans le domaine
		      de la traçabilité et du monitoring via la technologie RFID
		      \begin{itemize}
		      	\item \href{http://www.caves-explorer.com/}{Caves Explorer}~:
		      	      Logiciel de gestion de caves à vin grand public assisté
		      	      par des radio-étiquettes SLI, développé à l'aide du
		      	      framework Qt.
		      	      Synchronisation des données entre une base de données
		      	      distante MySQL et une base locale SQLite.
		      	      Solutions dérivés intégrés à la solution principale.
		      	\item \href{http://www.beetracking.com/}{Beetracking}~:
		      	      Solution de suivi de rucher et de marquage de ruche par
		      	      radio-étiquettes passives {UHF}.
		      	      Développement embarqué sur BOLYMIN, Armadeus et terminaux
		      	      portatifs durcis {PSION}.
		      	      Envois de SMS (commandes AT).
		      	\item
		      	      Développement d'une solution de traçabilité de chaîne de
		      	      production.
		      	      Système d'informations distribués.
		      	      Radio-étiquettes {MIFARE}.
		      \end{itemize}
		\item Développement d'une IHM en C/GTK pour gérer des données
		      vectorielles au sein d'une carte dans le cadre d'un projet destiné
		      à la Défense.
		\item Participation au développement d'applications Web.
	\end{itemize}}

\cventry{2010 \\
	\small\color{gray}{2 mois}}{Ouvier saisonnier}{R. Bourgeois}{Besançon}{Salarié}{%
Cerclage de palette}

\cventry{2010 \\
	\small\color{gray}{3 mois}}{Stage en entreprise}{Parkéon}{Besançon}{Stagiaire}{%
	Programmation en Java d'un logiciel chargé de piloter un banc d'encodage de
cartes magnétiques et de gérer les mappings de cartes}

\cventry{2009 \\
	\small\color{gray}{2 mois}}{Ouvrier d'exécution}{Saint Nazaire (Loire Atlantique)}{Renaissance SARL}{Salarié}{%
Travaux de rénovation, maroufflage de moteur de bateau}

\cventry{2008 \\
	\small\color{gray}{6 semaines}}{Stage en entreprise}{Besançon}{Selma (Aujourd'hui appelé \href{http://www.safed.ch/}{SAFED})}{Stagiaire}{%
Logiciel de monitoring de température de fours industriels}

\section{Centres d'intérêt}

\cvlistitem{Aikido}
\cvlistitem{Apiculture}

\end{document}
